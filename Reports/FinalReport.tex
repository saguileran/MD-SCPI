\documentclass[letter,10pt, twocolumn]{article}
\usepackage[utf8]{inputenc}
\usepackage{graphicx}
\usepackage[english]{babel} 
\usepackage{lipsum}
\usepackage{multicol}
\usepackage{geometry}
\geometry{
 left=20mm, right=20mm,
 top=20mm, bottom=20mm,
 }
 \usepackage{abstract}
\renewcommand{\abstractnamefont}{\normalfont\Large\bfseries}
\renewcommand{\abstracttextfont}{\normalfont\Huge}

\renewcommand{\thesection}{\Roman{section}} 
\renewcommand{\thesubsection}{\thesection.\Roman{subsection}}

\title{Evaluation of the unbinding kinetics of Mineralocorticoid (MR) receptor steroid agonist Cortisol (COL) , Aldosterone (AS4), and Progesteron (STR) ligands using Molecular Dynamics (MD) and Monte Carlo (MC) simulations}

\author{Sebastian Aguilera Novoa}
\date{April 14, 2023}



\begin{document}

%\maketitle

\twocolumn[
  \begin{@twocolumnfalse}
    \maketitle
    \begin{abstract}
    \normalsize
    asdasdasd
    \vspace{20pt}
    \end{abstract}
  \end{@twocolumnfalse}
]



\section{Introduction}

Why MR and these ligands

\section{Methods}

Selection of the best method to simulate the system, description of the simulation methods and how are we analyzing the date, since the MD simulations have a sense of time while the MC simulations does not, pyEMMA.



\section{Results and Discussion}

Select images of interest and explain them


\section{Conclusion}

What is possible to conclude from each simulation and how can it be explain from chemistry and physics

\begin{itemize}

\item 

\end{itemize}


\clearpage
\onecolumn{
\bibliography{../references}
\bibliographystyle{unsrt.bst} %Used BibTeX style is unsrt
\nocite{*}}

\end{document}
